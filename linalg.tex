
\documentclass[a4paper]{article}

\usepackage[danish]{babel}
\usepackage[utf8]{inputenc}
\usepackage{ulem}
\usepackage{gauss}
\usepackage{amssymb}

\title{Noter til Lineær Algebra}
\author{Daniel Schierbeck \and Lasse Caramés \and Andreas Hjordt Jensen}

% Matrix variables
\newcommand{\mtx}[1]{\uuline{#1}}

% Label format for \mult matrix operations
\renewcommand{\rowmultlabel}[1]{\times\,#1}

\renewcommand{\vec}[1]{\uline{#1}}

% Don't indent paragraphs
\setlength{\parindent}{0cm}

% The distance between a matrix and the first row operation
\setlength{\rowarrowsep}{0pt}

% Don't show heading numbers
\setcounter{secnumdepth}{0}

% Only show \section and \subsection headings in the ToC
\setcounter{tocdepth}{2}

\begin{document}

\maketitle

\begin{abstract}
Noter til faget Lineær Algebra ved Københavns Universitet, 2008.
\end{abstract}

\tableofcontents

\section{Matricer}

\subsection{Determinanter}

\subsubsection{$2 \times 2$ matrix}
$$
\mtx{A} =
\begin{gmatrix}[p]
    a_{11} & a_{12} \\
    a_{21} & a_{22}
\end{gmatrix}
$$

Determinanten af $\mtx{A}$ defineres som tallet
$$
\det{\mtx{A}} = a_{11}a_{22} - a_{12}a_{21}
$$


\subsubsection{$3 \times 3$ matrix}
$$
\mtx{A} =
\begin{gmatrix}[p]
    a_{11} & a_{12} & a_{13} \\
    a_{21} & a_{22} & a_{23} \\
    a_{31} & a_{32} & a_{33}
\end{gmatrix}
$$

Determinanten af $\mtx{A}$ defineres som tallet
$$
\det{\mtx{A}} = a_{11}a_{22}a_{33} + a_{12}a_{23}a_{31} + a_{13}a_{21}a_{32} - a_{13}a_{22}a_{31} - a_{11}a_{23}a_{32} - a_{12}a_{21}a_{33}
$$


\subsection{Matrixinverssion}
\subsubsection{Definition}
En matrix er regulær (eller invertibel) hvis den tilhørende lineære afbildning $f:\mathbb{R}^{n}\rightarrow\mathbb{R}^{n}$ er bijektiv.

For at finde den inverse skal det sikres at matricen er regulær.
\subsubsection{Regneregler}
\begin{enumerate}
\item $E^{-1}=E$
\item $(A^{-1})^{-1}=A$
\item $A^{-1}A=AA^{-1}=E$
\item $(AB)^{-1}=A^{-1}B^{ -1}$
\end{enumerate}
\subsubsection{Regulær?}

\subsection{Række- og søjleoperationer}

\subsubsection{Type M: multiplikation af en række med et tal $c \neq 0$}
$$
\begin{gmatrix}[p]
    1 & 2 & 3 \\
    4 & 5 & 6 \\
    7 & 8 & 9
\rowops
    \mult{0}{3}
\end{gmatrix} \longrightarrow
\begin{gmatrix}[p]
    3 & 6 & 9 \\
    4 & 5 & 6 \\
    7 & 8 & 9
\end{gmatrix}
$$


\subsubsection{Type B: ombytning af to rækker}
$$
\begin{gmatrix}[p]
    1 & 2 & 3 \\
    4 & 5 & 6 \\
    7 & 8 & 9
\rowops
    \swap{0}{1}
\end{gmatrix} \longrightarrow
\begin{gmatrix}[p]
    4 & 5 & 6 \\
    1 & 2 & 3 \\
    7 & 8 & 9
\end{gmatrix}
$$


\subsubsection{Type S: addition af et multiplum af en række til en anden række}
$$
\begin{gmatrix}[p]
    1 & 2 & 3 \\
    4 & 5 & 6 \\
    7 & 8 & 9
\rowops
    \add[2]{0}{1}
\end{gmatrix} \longrightarrow
\begin{gmatrix}[p]
    1 & 2 &  3 \\
    6 & 9 & 12 \\
    7 & 8 &  9
\end{gmatrix}
$$


\subsection{Diagonaliserbarhed}

\begin{itemize}
\item Matricen er symmetrisk.
\item Matricen er allerede diagonal.
\end{itemize}

\paragraph{Sætning 6.1.3}
Den lineære afbildning $f : V \rightarrow V$ er diagonaliserbar netop hvis der findes en basis for $V$ bestående af egenvektorer for $f$.


\section{Vektorrum}

\subsection{Egenvektor}

At $\vec{x}$ er en egenvektor for $f$ med tilhørende egenværdi $\lambda$ betyder, at
$$
\vec{x} \neq \vec{o} \quad \textrm{og} \quad f(\vec{x}) = \lambda{}\vec{x}
$$

\subsection{Baser}

\subsubsection{Række- og søjlerum}
En basis for rækkerummet er givet ved de rækker i rækkeechelonformen der ikke er nul-rækker.
En basis for søjlerummet findes ved at de ledende variable.


\end{document}
